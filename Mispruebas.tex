\documentclass{article}
\usepackage[utf8]{inputenc}
\usepackage[spanish]{babel}



\usepackage{graphicx} %paquete básico para incluir gráficos
\usepackage{wrapfig} %paquete Wrapfig
\usepackage{float} % paquete para controlar entornos flotantes
\usepackage{pdfpages}% Paquete pdfpages incluye páginas completas de ficheros pdfs
\usepackage{lipsum} % Paquete para introducir texto 
\title{Taller yosigopublicando}

\author{yo}
\date{\today}

\begin{document}

\maketitle

\section{Introduction}

\begin{abstract}
Esto es la continuación del curso básico de \LaTeX{} con overleaf.
\end{abstract}
%%%%%%%%%%%%%%%%%%%%%%%%%%%%%%%%%%%%%%%%%%%%%%%%%%%%%%
\section{Inclusión de Gráficos}















\end{document}