% Modelo de slides para projetos de disciplinas do Abel
\documentclass[10pt]{beamer}

\usetheme[progressbar=frametitle]{metropolis}
\usepackage{appendixnumberbeamer}
\usepackage[numbers,sort&compress]{natbib}
\bibliographystyle{plainnat}
\usepackage{booktabs}
\usepackage[scale=2]{ccicons}
\usepackage{xspace}
\newcommand{\themename}{\textbf{\textsc{metropolis}}\xspace}
%%%%%%%%%%%%%%%%%%%%%%%%%%%%%%%%%%%%%%%%%%%%
\usepackage[spanish]{babel}
% Paquete pdfpages incluye páginas completas de ficheros pdfs
\usepackage{pdfpages}
%TikZ 
\usepackage{pgf,tikz}
% PGFPlots
\usepackage{pgfplots}
\pgfplotsset{width=6cm,compat=1.12} %{compat=yourversion}
%paquete chemfig: gráficos de química
\usepackage{chemfig}
%paquete Wrapfig
\usepackage{wrapfig}

%%%%%%%%%%%%%%%%%%%%%%%%%%%%%%%
\title{Inserción/edición y creación \\ de 
gráficos con \LaTeX{}}
% \subtitle{Subtítulo}
% \date{\today}
\date{}
\author{Óscar Sánchez Romero}
\institute{Dpto. Matemática Aplicada, UGR}
% \titlegraphic{\hfill\includegraphics[height=1.5cm]{logo.pdf}}
%%%%%%%%%%%%%%%%%%%%%%%%%%%%%%%%%%%
\begin{document}

\maketitle

\begin{frame}{Contenidos}
  \setbeamertemplate{section in toc}[sections numbered]
  \tableofcontents[hideallsubsections]
\end{frame}

\section{Introducción}

\subsection{Uso básico}
\begin{frame}[fragile]{Uso básico}
Todos sabemos que en un documento generado con \LaTeX{} podemos incorporar  gráficos  
\begin{figure}[h]
\includegraphics[width=0.3\textwidth]{graficos/sorpresa.jpg} 
\includegraphics[width=0.3\textwidth]{graficos/fig_9Vis.pdf}
\includegraphics[width=0.3\textwidth]{graficos/ciclista.png}
\caption{Caption general}
\label{figGeneral}
\end{figure}
\end{frame}
%%%%%%%%%%%%%%%%%%%%%%%%%%%%%%%%%%%%%%%%%%%%
\subsection{Uso no tan básico}
\begin{frame}[fragile]{Uso no tán básico}
Lo que no es tan conocido es que, al incorporarlos, permite editarlos ligeramente.
%\vspace{-1cm}
%\includegraphics[angle=0, width=3cm]{./graficos/ciclista}
\begin{figure}
\centering
\includegraphics[width=6cm]{./graficos/fig_9Vis}
\put(-90,5){\includegraphics[angle=-10,scale=0.4]{./graficos/ciclista}}
\caption{Redimensionar, girar y superponer imágenes}
\end{figure}
{\scriptsize OJO!  Requiere 
\href{https://docs.gimp.org/2.10/es/gimp-tutorial-quickie-separate.html}{fondo transparente} en la figura a superponer (objeto).}
\end{frame}
%%%%%%%%%%%%%%%%%%%%%%%%%%%%%%%%%%%%%%%%%%%%%%

\begin{frame}[fragile]{Uso no tán básico}
Lo que no es tan conocido es que, al incorporarlos, permite editarlos ligeramente.

\begin{figure}
\includegraphics[trim = 50mm 0mm 190mm 40mm, clip,width=4.5cm]{./graficos/sorpresa}
\hspace{-0.3cm}
\reflectbox{
\includegraphics[trim = 50mm 0mm 190mm 40mm, clip,width=4.5cm]{./graficos/sorpresa}
}
\put(-195,135){{\color{green}   \LARGE \textbf{Powered by \LaTeX}}}
\caption{Selección, simetrizar, incluir texto}
\end{figure}

\end{frame}
%%%%%%%%%%%%%%%%%%%%%%%%%%%%
\begin{frame}[fragile]{Otra utilidad: inclusión de partes de ficheros pdf}
\vspace{-4cm}
\begin{wrapfigure}{r}{6cm}
\hspace{0.5cm}
\includegraphics[width=5.cm]{graficos/Diploma.pdf}
\end{wrapfigure}
\vspace{1cm}
El paquete pdfpages \cite{pdfpages} permite incluir páginas seleccionadas de un pdf en un documento \LaTeX{}.

{\scriptsize Utilidad: generar documentación acreditativa.}

\begin{verbatim}
\includepdf[]{file.pdf}
\end{verbatim}
\end{frame}
%%%%%%%%%%%%%%%%%%%%%%%%%%%%%%
\subsection{Uso avanzado}
\begin{frame}[fragile]{Uso avanzado: paquetes específicos}
Otra posibilidad que ofrece \LaTeX{} es generar con él los gráficos.
Un primer ejemplo puede ser \href{http://pgfplots.net}{pgfplots} para representar funciones y datos.
\begin{wrapfigure}{r}{5cm}
\begin{tikzpicture} \begin{axis}[
    title={$x \exp(-x^2-y^2)$},
    xlabel=$x$, ylabel=$y$,
    small,
] \addplot3[
    surf,
    domain=-2:2,
    domain y=-1.3:1.3,
]
    {exp(-x^2-y^2)*x};
\end{axis}
\end{tikzpicture}
\caption{Gr\'afico tridimensional}
\end{wrapfigure}
$ $ %\vspace{1cm}
\begin{verbatim}
\begin{tikzpicture} 
\begin{axis}[
    title={$x \exp(-x^2-y^2)$},
    xlabel=$x$, ylabel=$y$,
    small] 
\addplot3[surf,
    domain=-2:2,
    domain y=-1.3:1.3,
]{exp(-x^2-y^2)*x};
\end{axis}
\end{tikzpicture}
\end{verbatim}
\end{frame}
%%%%%%%%%%%%%%%%%%%%%%%%%%%%%%
\begin{frame}[fragile]{Uso avanzado: paquetes específicos}
Este uso se está extendiendo en \'areas distintas a la Física o las Matemática como por ejemplo el paquete chemfig \cite{Chemical} en \href{https://www.ctan.org/search/index?phrase=chemistry&offset=48&max=16}{Química}
\begin{wrapfigure}{r}{4cm}
\chemfig{-(-[1]O^{\ominus})=[7]O}

\chemfig{C(-[:0]H)(-[:90]H)(-[:180]H)(-[:270]H)}
\caption{Gr\'aficos química}
\end{wrapfigure}
\begin{verbatim}
\chemfig{-(-[1]O^{\ominus})=[7]O}
\end{verbatim}
\vspace{1cm}
\begin{verbatim}
\chemfig{C(-[:0]H)(-[:90]H)
(-[:180]H)(-[:270]H)}
\end{verbatim}
\end{frame}

%%%%%%%%%%%%%%%%%%%%%%%%%%%%%%%%%%%%%%%%%%%%%%%%%%%%%%%%%%%%%%%%%%%%%
\begin{frame}{Uso avanzado: creación con Tikz}
Figura generada directamente con comandos \href{http://www.texample.net/tikz/examples/}{Tikz}
\begin{figure}
    \centering
    \includegraphics[scale=0.8]{graficos/bp2.pdf}
    \caption{Figura generada con Tikz}
    \label{figuraTikz}
\end{figure}
\end{frame}

%%%%%%%%%%%%%%%%%%%%%%%%%%%%%%
\begin{frame}[fragile]{Uso avanzado: sitios de interés}
Puesto que es imposible mostrar paquetes de interés para una 
audiencia heterogénea lo mejor es mostrar dónde y cómo localizarlos

\href{https://www.ctan.org}{https://www.ctan.org}

\href{https://es.wikibooks.org/wiki/Manual_de_LaTeX/Inclusión_de_gráficos/Gráficos_con_TikZ}{Wikibooks: Graficos con Tikz}

\href{http://www.texample.net/tikz/examples/}{http://www.texample.net/tikz/examples/}

\end{frame}


%%%%%%%%%%%%%%%%%%%%%%%%%%%%
\section{Preparación de gráficos}
\begin{frame}{Generalidades sobre formatos gr\'aficos}
\begin{block}{Mapas de bits} 
\begin{center}
\includegraphics[width=7cm]{graficos/sorpresa.jpg}
\end{center}
Extensiones: BMP, JPEG, GIF, PNG y TIFF. \\
{\small Desventaja: deformaciones al reescalar y gran tama\~no.}
\end{block}
\end{frame}
%%%%%%%%%%%%%%%%%%%%%%%%%%%%
\begin{frame}{Generalidades sobre formatos gr\'aficos}
\begin{block} {Gr\'aficos vectoriales} 
%\vspace{-1cm}
\begin{center}
\includegraphics[width=10cm]{graficos/bp2.pdf}
\end{center}
Extensiones: EPS, PDF, SVG, WMF \\
{\small Nota: !`Estos archivos pueden insertar mapas de bits! }
\end{block}
\end{frame}
%%%%%%%%%%%%%%%

\begin{frame}{Preparaci\'on de gr\'aficos para insertar en \LaTeX}
El formato del gr\'afico a insertar depende del compilador empleado:
\begin{enumerate}
\item \it{latex + dvips}  se requiere PS / EPS {\scriptsize(con \href{http://tex.stackexchange.com/questions/133786/no-boundingbox-error-message}{BoundingBox})}
\item \it {pdflatex} se requiere PNG {\scriptsize(mapas de bits simples)}, JPEG {\scriptsize(fotograf\'ias)} o PDF {\scriptsize(gr\'aficos vectoriales)}
\end{enumerate}

\vspace{0.5cm}
Esto requiere de programas espec\'ificos de transformaci\'on:
\begin{itemize}
\item {\sc EPS a PDF}: \href{http://tug.org/epstopdf/}{epstopdf}
%\item {\sc JPEG a EPS}: \href{http://www.pdflib.com/download/free-software/jpeg2ps/}{jpeg2ps}
\item {\sc Todo a Todo}: \href{http://www.inkscape.org/es/}{Inkscape}, 
\href{http://www.imagemagick.org}{ImageMagick} o \href{http://www.gimp.org/}{Gimp}
\item ........
\end{itemize}
Ver detalles en \cite{ManualLatexWikilibros}.
\end{frame}

%%%%%%%%%%%%%%%%%%%%%%%%%%%%%%%%%%%%%%%%%%
\section{Inserci\'on/edici\'on de un gr\'afico}
%%%%%%%%%%%%%%%%%%%%%
\begin{frame}[fragile]{Insertar el gr\'afico como una figura}
Declaraci\'on del paquete graphicx en el pre\'ambulo:
\begin{verbatim}\usepackage{graphicx} \end{verbatim}

Inserci\'on del gr\'afico en el documento:
\begin{verbatim}
\begin{figure}
  \centering
  \includegraphics[parametros]{nombregrafico}
  \caption{Leyenda bajo el grafico}
  \label{fig:etiqueta}
\end{figure}
\end{verbatim}
Mediante los par\'ametros se puede modificar el aspecto, lo que 
nos permite editarlos ligeramente.
\end{frame}

%%%%%%%%%%%%%%%%%%%%%%%%%%%%%%%%%%%%%%%%%%
\begin{frame}[fragile]{Parámetros para modificar una figura}
Parámetros empleados más usualmente:
\begin{itemize}
    \item \verb|scale=0.5 | \hfill escala el tamaño a la mitad
    \item \verb|height=5cm | \hfill fija la anchura del gráfico a 5cm
    \item \verb|width=0.5\textwidth | \hfill anchura = mitad del espacio para texto.
    \item \verb|angle=90 |\hfill  gira la imagen 90 grados.
    \item \verb|trim = 10mm 5mm 50mm 55mm, clip| \hfill Recorta la imagen quitando
\verb|trim = <left> <lower> <right> <upper> | \hfill 10mm por izda,...
    \item \verb|draft| \hfill no se incluye el gráfico pero deja el espacio apropiado.
\end{itemize}
\vspace{1cm}
Para profundizar ver documentaci\'on paquete graphicx \cite{DocGraphicx} .
\end{frame}
%%%%%%%%%%%%%%%%%%%%%%%%%%%%%%%%%%%%%%%%%%
\begin{frame}[fragile]{Localización de la figura en el texto}
El entorno figure es flotante, esto es, \LaTeX{} ``decide''
dónde lo pone. Si queremos controlar este proceso tenemos varias 
opciones:
\begin{itemize}
    \item Usar \verb|includegraphics| sin entorno \verb|figure|. \hfill (No recomendable)
    \item Control débil del entorno \verb|figure| con parámetros 
    de control h, b, t. 
    \item Empleo del entorno \verb|wrapfigure| \cite{wrapfigure} gracias al paquete \verb|wapfig|.
    \item Empleo del parámetro H del paquete \verb|float|.
\end{itemize}
\vspace{1cm}
Ver detalles en \href{https://www.overleaf.com/learn/latex/Positioning_images_and_tables}{ayuda de OverLeaf}.
\end{frame}
%%%%%%%%%%%%%%%%%%%%%%%%%%%%%%%%%%%%%%%%%%%%%%%
\begin{frame}[allowframebreaks]{Referencias}
\bibliography{presentacion}
%\bibliographystyle{abbrv}
\end{frame}

\end{document}
